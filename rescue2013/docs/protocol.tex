\documentstyle{article}
\title{Robocup Rescue Simulation - Communication Protocol}
\author{Cameron Skinner\\Department of Computer Science\\The University of Auckland}
\begin{document}
\maketitle
\section{Introduction}
This protocol describes the data exchange between the kernel, gis, simulators and agents in the Robocup Rescue Simulation software. The protocol has two layers, the transport layer and the content layer.

All numbers in both layers are encoded with the most significant byte first (big-endian).

\section{Transport Layer}
The transport layer is responsible for sending a \textit{message} (i.e. a number of bytes) between simulator components. There are two transport mechanisms at present: UDP/IP or TCP/IP. The sending of a message may or may not block. Implementers are encouraged to provide both blocking and non-blocking implementations of the transport layer.

\subsection{UDP}
UDP packets may not be larger than 65536 bytes. Messages that are sent over UDP might be larger than this limit, so the UDP layer is wrapped in the \textit{LongUDP} protocol.

LongUDP breaks the message into pieces and adds an 8 byte header to each piece. The size of the pieces is arbitrary but is guaranteed to be less than 65528. Each message is assigned a unique ID and each piece of the message is assigned a sequence number. Sequence numbers start from zero. All values are unsigned.

Note that UDP does not guarantee delivery so partial packets may result from the use of LongUDP. These should be discarded after a suitable timeout.

The 8 byte header is as shown in Table \ref{table_longudp_header}.

\begin{table}[p]
\center
\begin{tabular}{|l|l|l|}
\hline
Byte Offset & Value & Meaning\\
\hline
0 & 0x0008 & Magic number\\
2 & ID & The ID of the message\\
4 & Sequence number & The ID of this piece of the message\\
6 & Total & The total number of pieces\\
\hline
\end{tabular}
\caption{The LongUDP header}
\label{table_longudp_header}
\end{table}

Once all \textit{total} pieces have been received the data segments can be concatenated in order of sequence number to reconstruct the original message.

\subsection{TCP}
The TCP protocol sends the data prefixed by a 32-bit integer specifying the length of the message.

\section{Content Layer}
The content layer describes the format of the messages that are exchanged between simulator components.

There are several types of data that can be encoded. Integers are encoded big-endian. Lists of integers are encoded by writing the length of the list as a 32-bit integer followed by the content of the list.

Properties are encoded as shown in Table \ref{table_property_format}. Lists of properties are zero-terminated. Similarly, objects are encoded as shown in Table \ref{table_object_format} and lists of objects are zero terminated. It is possible to encode only those properties that have changed, when appropriate. Implementors are encouraged to provide functions for writing all properties and for writing only those properties that have been modified.

An example of a list of one object is shown in Figure \ref{figure_example_object}.

Strings are encoded in ASCII and prefixed with the length of the string in bytes. Figure \ref{figure_example_string} has an example of this encoding.

The list of possible object types is shown in Table \ref{table_object_types}. The list of property types is shown in Tables \ref{table_int_property_types} and \ref{table_int_list_property_types}.

\begin{table}[p]
\center
\begin{tabular}{|l|l|l|}
\hline
Field & Data type & Meaning\\
\hline
Header & int & Property type\\
Length & int & Size of property data in bytes\\
Value & int \textbf{OR} int list & Property value\\
\hline
\end{tabular}
\caption{Property encoding format}
\label{table_property_format}
\end{table}

\begin{table}[p]
\center
\begin{tabular}{|l|l|l|}
\hline
Field & Data type & Meaning\\
\hline
Header & int & Object type\\
Length & int & Size of object data in bytes\\
ID & int & Object ID\\
Data & Property list & Properties of this object\\
\hline
\end{tabular}
\caption{Object encoding format}
\label{table_object_format}
\end{table}

\begin{figure}[p]
\center
\begin{tabular}{|l|l|l|}
\hline
Byte Offset & Value & Meaning\\
\hline
0 & 0xE8 & Object header: TYPE\_CIVILIAN\\
4 & 0x22 & Length of object data (40 bytes)\\
8 & 0x10 & ID of the object\\
12 & 0x06 & Property header: POSITION\\
16 & 0x04 & Length of property data (4 bytes)\\
20 & 0xAABBCCDD & Value of property\\
24 & 0x07 & Property header: POSITION\_HISTORY\\
28 & 0x0C & Length of property data (12 bytes)\\
32 & 0x02 & Number of entries\\
36 & 0x11223344 & Entry 1\\
40 & 0x55667788 & Entry 2\\
44 & 0x00 & Property header: NULL\\
48 & 0x00 & Object header: NULL\\
\hline
\end{tabular}
\caption{An example object encoding}
\label{figure_example_object}
\end{figure}

\begin{figure}[p]
\center
\begin{tabular}{|l|l|l|}
\hline
Byte Offset & Value & Meaning\\
\hline
0 & 0x0D & Length of the string\\
4 & 0x48 & H\\
5 & 0x65 & e\\
6 & 0x6c & l\\
7 & 0x6c & l\\
8 & 0x6f & o\\
9 & 0x20 &  \\
10 & 0x57 & W\\
11 & 0x6f & o\\
12 & 0x72 & r\\
13 & 0x6c & l\\
14 & 0x64 & d\\
15 & 0x21 & !\\
\hline
\end{tabular}
\caption{An example string encoding}
\label{figure_example_string}
\end{figure}

\subsection{Commands}
Commands consist of a 32-bit header describing the type of the command, a 32-bit integer containing the size of the command in bytes followed by the content of the command. Commands can be concatenated into a zero-terminated list. An example of a list of two fictional commands is shown in Table \ref{table_fake_command_list}. When encoding objects, generally only those properties that have been changed will be written.

\begin{table}[p]
\center
\begin{tabular}{|l|l|l|}
\hline
Byte Offset & Value & Meaning\\
\hline
0 & 0x99 & Command type (fictional command ``99'')\\
4 & 4 & Length in bytes\\
8 & 1234 & Data \\
12 & 0x99 & Command type (another fictional ``99'' command)\\
16 & 8 & Length in bytes \\
20 & 4321 & Data\\
24 & 8765 & Data\\
28 & 0 & The list is zero-terminated\\
\hline
\end{tabular}
\caption{A list of two fictional commands}
\label{table_fake_command_list}
\end{table}

The list of possible command types and their header values is shown in Table \ref{table_commands} and is broken down into those that concern the GIS, simulators and agents.

\subsubsection{GIS Commands}
When the kernel starts it needs to connect to the GIS with a KG\_CONNECT command. The GIS replies with either a GK\_CONNECT\_OK or GK\_CONNECT\_ERROR. The kernel replies with a KG\_ACKNOWLEDGE if the connection was successful. The formats of these messages are shown in Table \ref{table_gis_commands}

\begin{table}[p]
\center
\begin{tabular}{|lll|}
\hline
Data type & Meaning & Notes\\
\hline
\hline
\multicolumn{3}{|c|}{KG\_CONNECT}\\
int & Version & Unused\\
\hline
\hline
\multicolumn{3}{|c|}{GK\_CONNECT\_OK}\\
Object list & The objects in the world & \\
\hline
\hline
\multicolumn{3}{|c|}{GK\_CONNECT\_ERROR}\\
string & The reason for the error & \\
\hline
\hline
\multicolumn{3}{|c|}{KG\_ACKNOWLEDGE}\\
\hline
\end{tabular}
\caption{GIS commands}
\label{table_gis_commands}
\end{table}

\subsubsection{Viewer Commands}
Viewers connect to the kernel with a VK\_CONNECT command. The kernel replies with KV\_CONNECT\_OK or KV\_CONNECT\_ERROR and the viewer acknowledges a successful connection with VK\_ACKNOWLEDGE. The message formats are shown in Table \ref{table_viewer_commands}.

\begin{table}[p]
\center
\begin{tabular}{|lll|}
\hline
Data type & Meaning & Notes\\
\hline
\hline
\multicolumn{3}{|c|}{VK\_CONNECT}\\
int & Version & Unused\\
\hline
\hline
\multicolumn{3}{|c|}{KV\_CONNECT\_OK}\\
Object list & The objects in the world & \\
\hline
\hline
\multicolumn{3}{|c|}{KV\_CONNECT\_ERROR}\\
string & The reason for the error & \\
\hline
\hline
\multicolumn{3}{|c|}{VK\_ACKNOWLEDGE}\\
\hline
\end{tabular}
\caption{Viewer commands}
\label{table_viewer_commands}
\end{table}

\subsubsection{Simulator Commands}
Simulators connect to the kernel with an SK\_CONNECT command. The kernel replies with KS\_CONNECT\_OK or KS\_CONNECT\_ERROR and the simulator acknowledges a successful connection with SK\_ACKNOWLEDGE. Each timestep the simulators will receive a COMMANDS message (described in section \ref{section_broadcast_commands}) and must reply with an SK\_UPDATE. Once all simulators have replied the kernel will send an UPDATE message (also described in section \ref{section_broadcast_commands}). The message formats are shown in Table \ref{table_simulator_commands}.

\begin{table}[p]
\center
\begin{tabular}{|lll|}
\hline
Data type & Meaning & Notes\\
\hline
\hline
\multicolumn{3}{|c|}{SK\_CONNECT}\\
int & Version & Unused\\
\hline
\hline
\multicolumn{3}{|c|}{KS\_CONNECT\_OK}\\
int & The ID of the simulator & \\
Object list & The objects in the world & \\
\hline
\hline
\multicolumn{3}{|c|}{KS\_CONNECT\_ERROR}\\
string & The reason for the error & \\
\hline
\hline
\multicolumn{3}{|c|}{SK\_ACKNOWLEDGE}\\
int & The ID of the simulator & Taken from KS\_CONNECT\_OK \\
\hline
\hline
\multicolumn{3}{|c|}{SK\_UPDATE}\\
int & The ID of the simulator & Taken from KS\_CONNECT\_OK \\
int & Time & \\
Object list & Changed objects & \\
\hline
\end{tabular}
\caption{Simulator commands}
\label{table_simulator_commands}
\end{table}

\subsubsection{Agent Commands}
Agents connect to the kernel with an AK\_CONNECT command. The kernel replies with KA\_CONNECT\_OK or KA\_CONNECT\_ERROR and the agent acknowledges a successful connection with AK\_ACKNOWLEDGE. The message formats are shown in Tables \ref{table_agent_commands} and \ref{table_more_agent_commands}.

\begin{table}[p]
\center
\begin{tabular}{|lll|}
\hline
Data type & Meaning & Notes\\
\hline
\hline
\multicolumn{3}{|c|}{AK\_CONNECT}\\
int & Temporary ID & Unique temporary ID\\
int & Version & Unused\\
int & Agent type & Logical OR of requested types\\
\hline
\hline
\multicolumn{3}{|c|}{KA\_CONNECT\_OK}\\
int & Temporary ID & Taken from AK\_CONNECT\\
int & The real ID of the agent & Assigned by kernel\\
Object & Self & Object controlled by this agent\\
Object list & The objects in the world & \\
\hline
\hline
\multicolumn{3}{|c|}{KA\_CONNECT\_ERROR}\\
int & Temporary ID & Taken from AK\_CONNECT\\
string & The reason for the error & \\
\hline
\hline
\multicolumn{3}{|c|}{AK\_ACKNOWLEDGE}\\
int & Agent ID & Taken from KA\_CONNECT\_OK \\
\hline
\hline
\multicolumn{3}{|c|}{KA\_SENSE}\\
int & Agent ID & \\
int & Time & \\
Object list & All changed objects & \\
\hline
\hline
\multicolumn{3}{|c|}{KA\_HEAR}\\
int & Agent ID & \\
int & ID of sender & \\
int & Channel number & \\
int & Message length & \\
byte array & The message & \\
\hline
\end{tabular}
\caption{Agent commands}
\label{table_agent_commands}
\end{table}

\begin{table}[p]
\center
\begin{tabular}{|lll|}
\hline
Data type & Meaning & Notes\\
\hline
\hline
\multicolumn{3}{|c|}{AK\_REST}\\
int & Agent ID & \\
\hline
\hline
\multicolumn{3}{|c|}{AK\_MOVE}\\
int & Agent ID & \\
int list & Path to be moved & \\
\hline
\hline
\multicolumn{3}{|c|}{AK\_EXTINGUISH}\\
int & Agent ID & \\
nozzle list & List of nozzles to be used & \\
\multicolumn{3}{|l|}{\hspace{5em}Nozzle format}\\
int & Target ID & \\
int & Target direction & Unused\\
int & Nozzle x coordinate & Unused\\
int & Nozzle y coordinate & Unused\\
int & Amount of water & \\
\hline
\hline
\multicolumn{3}{|c|}{AK\_LOAD}\\
int & Agent ID & \\
int & Target ID & \\
\hline
\hline
\multicolumn{3}{|c|}{AK\_UNLOAD}\\
int & Agent ID & \\
\hline
\hline
\multicolumn{3}{|c|}{AK\_RESCUE}\\
int & Agent ID & \\
int & Target ID & \\
\hline
\hline
\multicolumn{3}{|c|}{AK\_CLEAR}\\
int & Agent ID & \\
int & Target ID & \\
\hline
\hline
\multicolumn{3}{|c|}{AK\_TELL}\\
int & Agent ID & \\
int & Target channel &  $\in [0,255]$\\
int & Message length & \\
byte array & Message body & \\
\hline
\hline
\multicolumn{3}{|c|}{AK\_CHANNEL}\\
int & Agent ID & \\
int & Desired channels length & \\
byte array & Desired channels & \\
\hline
\end{tabular}
\caption{Agent commands (continued)}
\label{table_more_agent_commands}
\end{table}

\subsubsection{Broadcast Commands}
\label{section_broadcast_commands}
Each timestep the kernel collects all agent commands and broadcasts them to simulators, viewers and log files via a COMMANDS messages. Similarly, updates from simulators are broadcast to simulators, viewers and log files with an UPDATE message. These messages are described in Table \ref{table_broadcast_commands}.

\begin{table}[p]
\center
\begin{tabular}{|lll|}
\hline
Data type & Meaning & Notes\\
\hline
\hline
\multicolumn{3}{|c|}{COMMANDS}\\
int & Time & \\
Command list & List of all commands & \\
\hline
\hline
\multicolumn{3}{|c|}{UPDATE}\\
int & Time & \\
Object list & All changed objects & \\
\hline
\end{tabular}
\caption{Broadcast commands}
\label{table_broadcast_commands}
\end{table}

\clearpage
\renewcommand{\thesection}{Appendix \Alph{section}}
\setcounter{section}{0}
\section{Tables of Constants}

\begin{table}[pht]
\center
\begin{tabular}{|l|l|}
\hline
Type & Value\\
\hline
WORLD & 0x01 \\
ROAD & 0x02 \\
RIVER & 0x03 \\
NODE & 0x04 \\
RIVERNODE & 0x05 \\
BUILDING & 0x20 \\
REFUGE & 0x21 \\
FIRE\_STATION & 0x22 \\
AMBULANCE\_CENTER & 0x23 \\
POLICE\_OFFICE & 0x24 \\
CIVILIAN & 0x40 \\
CAR & 0x41 \\
FIRE\_BRIGADE & 0x42 \\
AMBULANCE\_TEAM & 0x43 \\
POLICE\_FORCE & 0x44 \\
\hline
\end{tabular}
\caption{All possible object types}
\label{table_object_types}
\end{table}

\begin{table}[pht]
\center
\begin{tabular}{|lc|}
\hline
Type & Value\\
\hline
START\_TIME & 1 \\
LONGITUDE & 2 \\
LATITUDE & 3 \\
WIND\_FORCE & 4 \\
WIND\_DIRECTION & 5 \\

HEAD & 6 \\
TAIL & 7 \\
LENGTH & 8 \\

ROAD\_KIND & 9 \\
CARS\_PASS\_TO\_HEAD & 10 \\
CARS\_PASS\_TO\_TAIL & 11 \\
HUMANS\_PASS\_TO\_HEAD & 12 \\
HUMANS\_PASS\_TO\_TAIL & 13 \\
WIDTH & 14 \\
BLOCK & 15 \\
REPAIR\_COST & 16 \\
MEDIAN\_STRIP & 17 \\
LINES\_TO\_HEAD & 18 \\
LINES\_TO\_TAIL & 19 \\
WIDTH\_FOR\_WALKERS & 20 \\
SIGNAL & 21 \\

X & 25 \\
Y & 26 \\

FLOORS & 28 \\
BUILDING\_ATTRIBUTES & 29 \\
IGNITION & 30 \\
FIERYNESS & 31 \\
BROKENNESS & 32 \\
BUILDING\_CODE & 34 \\
BUILDING\_AREA\_GROUND & 35 \\
BUILDING\_AREA\_TOTAL & 36 \\

POSITION & 38 \\
POSITION\_EXTRA & 39 \\
DIRECTION & 40 \\
STAMINA & 42 \\
HP & 43 \\
DAMAGE & 44 \\
BURIEDNESS & 45 \\
WATER\_QUANTITY & 46 \\
\hline
\end{tabular}
\caption{All property types that have integer data}
\label{table_int_property_types}
\end{table}

\begin{table}[pht]
\center
\begin{tabular}{|lc|}
\hline
Type & Value \\
\hline
SHORTCUT\_TO\_TURN & 22 \\
POCKET\_TO\_TURN\_ACROSS & 23 \\
SIGNAL\_TIMING & 24 \\
EDGES & 27 \\
ENTRANCES & 33 \\
POSITION\_HISTORY & 41 \\
\hline
\end{tabular}
\caption{All property types that have lists of integer data}
\label{table_int_list_property_types}
\end{table}

\begin{table}[pht]
\center
\begin{tabular}{|l|l|l|}
\hline
Command & Header & Use\\
\hline
KG\_CONNECT & 0x10 & Kernel connects to GIS\\
KG\_ACKNOWLEDGE & 0x11 & Kernel acknowledges connection\\
GK\_CONNECT\_OK & x012 & GIS accepts kernel connection\\
GK\_CONNECT\_ERROR & 0x13 & GIS rejects kernel connection\\
\hline
SK\_CONNECT & 0x20 & Simulator connects to kernel\\
SK\_ACKNOWLEDGE & 0x21 & Simulator acknowledges connection\\
SK\_UPDATE & 0x22 & Simulator sends update to kernel\\
KS\_CONNECT\_OK & 0x23 & Kernel accepts simulator connection\\
KS\_CONNECT\_ERROR & 0x24 & Kernel rejects simulator connection\\
\hline
VK\_CONNECT & 0x30 & Viewer connects to kernel\\
VK\_ACKNOWLEDGE & 0x31 & Viewer acknowledges connection\\
KV\_CONNECT\_OK & 0x32 & Kernel accepts viewer connection\\
KV\_CONNECT\_ERROR & 0x33 & Kernel rejects viewer connection\\
\hline
AK\_CONNECT & 0x40 & Agent connects to kernel\\
AK\_ACKNOWLEDGE & 0x41 & Agent acknowledges connection\\
KA\_CONNECT\_OK & 0x42 & Kernel accepts agent connection\\
KA\_CONNECT\_ERROR & 0x43 & Kernel rejects agent connection\\
KA\_SENSE & 0x44 & Kernel sends update to agent\\
KA\_HEAR & 0x45 & Kernel sends an audio input to agent\\
\hline
UPDATE & 0x50 & Kernel broadcasts an update\\
COMMANDS & 0x51 & Kernel broadcasts agent commands\\
\hline
AK\_REST & 0x80 & Agent does nothing\\
AK\_MOVE & 0x81 & Agent moves\\
AK\_LOAD & 0x82 & Agent loads a victim\\
AK\_UNLOAD & 0x83 & Agent unloads a victim\\
AK\_TELL & 0x85 & Agent sends a message\\
AK\_EXTINGUISH & 0x86 & Agent extinguishes a fire\\
AK\_RESCUE & 0x88 & Agent rescues a buried victim\\
AK\_CLEAR & 0x89 & Agent clears a blocked road\\
AK\_CHANNEL & 0x90 & Agent listens to channels\\
\hline
\end{tabular}
\caption{All commands}
\label{table_commands}
\end{table}


\end{document}